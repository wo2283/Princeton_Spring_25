\documentclass[a4paper,12pt]{report}
\usepackage[a4paper, margin=1in]{geometry}
\usepackage{amsmath, amssymb, amsthm}
\usepackage{hyperref}
\usepackage{xcolor}
\usepackage{enumitem}
\usepackage{fancyhdr}
\usepackage{tcolorbox}
\usepackage{titlesec}

% Header/Footer
\pagestyle{fancy}
\fancyhf{}
\fancyhead[L]{\textbf{Course Name}}  % Change this to your course
\fancyhead[C]{\textbf{Lecture Notes}}
\fancyhead[R]{\textbf{\thepage}}

% Theorem & Definition Formatting
\theoremstyle{definition}
\newtheorem{definition}{Definition}[section]
\newtheorem{theorem}{Theorem}[section]

% Custom Highlight Box
\newtcolorbox{summarybox}{colback=blue!5!white, colframe=blue!50!black, 
boxrule=0.8pt, arc=5pt, left=5pt, right=5pt, top=5pt, bottom=5pt}

% Title
\title{\textbf{ORF307: Optimization \\ Lecture Notes}}
\author{William Oh}
\date{\today}

\begin{document}

\maketitle
\tableofcontents % Generates an automatic table of contents

% --- Lecture 1 ---
\chapter{Lecture 1: Introduction to Topic}
\section{Key Concepts}
\begin{definition}
    A \textbf{set} is a collection of distinct objects, considered as an object in its own right.
\end{definition}

\begin{theorem}[Pythagorean Theorem]
    In a right-angled triangle:
    \[
    a^2 + b^2 = c^2
    \]
\end{theorem}

% --- Lecture 2 ---
\chapter{Lecture 2: Advanced Concepts}
\section{Important Theorems}
\begin{summarybox}
    The derivative of \( f(x) = x^n \) is:
    \[
    \frac{d}{dx} x^n = n x^{n-1}
    \]
\end{summarybox}

\section{Examples}
\begin{itemize}
    \item Compute \( \int x^2 dx \).
    \item Find the eigenvalues of \( A = \begin{bmatrix} 2 & 1 \\ 1 & 2 \end{bmatrix} \).
\end{itemize}

\end{document}
